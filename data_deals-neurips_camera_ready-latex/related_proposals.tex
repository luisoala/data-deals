The call for a more equitable data value chain is not new. Our proposal for an \sysname builds upon, synthesizes, and extends several lines of existing work spanning conceptual frameworks, organizational models, technical implementations, and specific regulatory mechanisms.


\textbf{Conceptual foundations: Data dignity, labor, and dividends}. The philosophical underpinning of \sysname resonates deeply with concepts like ``Data Dignity'' and ``Data as Labor,'' notably championed by Jaron Lanier~\cite{LanierWeylNYT2019} and E. Glen Weyl~\cite{posner2018radical}. These frameworks argue for recognizing the economic value of individual data contributions and advocate for systems where originators are compensated, aligning with \sysname's core goals of traceable and equitable revenue-sharing. Similarly, proposals for ``Data Dividends'' (e.g., \cite{DataDividendFAQ}), distributing profits from data back to contributors. While these ideas provide powerful normative grounding, \sysname seeks to translate them into an actionable technical and economic protocol.

\textbf{Organizational models for collective bargaining}. To address the ``asymmetric bargaining power'' we identify, various organizational models have been proposed. Data Cooperatives enable members to pool data for collective negotiation and shared benefits (e.g.,  MIDATA in healthcare~\cite{MidataCoopHome}). This goal of empowering data subjects through collective action is also central to Delacroix and Lawrence's exploration of bottom-up ``data Trusts''~\cite{delacroix2019bottom}, and is pursued through collective bargaining mechanisms in what Freedman discusses as data unions~\cite{freedman2023data}. Building on this, this paper discusses mechanisms to facilitate the creation of dynamic data unions. These are specifically designed to empower data contributors—particularly those not part of established collectives—by enabling them to form strategically around the requirements of specific ML tasks, thereby enhancing their capacity to influence ML model development and strengthen their bargaining power.

\textbf{Online data marketplaces}. Online data markets aim to enable data deals at scale, but they face significant hurdles regarding data pricing and data quality. Traditional data marketplaces~\cite{aws_data_exchange,databricks_marketplace,narrative,taus,gradient,snowflake_datamarket}, for instance, typically utilize one-time upfront fees, query-based pricing, or subscriptions, which inadequately capture the context-dependent value of data, especially for machine learning applications. These platforms also offer limited information for potential buyers to evaluate dataset suitability, such as limited samples and metadata, making the process of finding suitable, high-quality data inefficient and uncertain. Recent efforts in the Web3 space~\cite{oceanprotocol_website,masa_ai_website,saharalabs_ai_website,bittensor} have focused on creating decentralized data marketplaces, aimed to enhance transparency, introduce token-based compensation, and broaden participation by enabling more, often smaller, players to engage in data transactions. However, they often grapple with the same fundamental challenges of valuation and data quality, which the envisioned \sysname addresses.

\textbf{Regulatory frameworks}. The evolving regulatory landscape increasingly recognizes the need for fairness and transparency in data handling, particularly with the rise of AI. Landmark regulations like the EU's General Data Protection Regulation (GDPR)~\cite{regulation2018general} 
have established strong protections for personal data, emphasizing consent, data subject rights, and accountability. More recently, the EU Data Act~\cite{EUDataAct2023} aims to ensure fairness in the allocation of data value in the digital economy, granting users (both individuals and businesses) greater rights to access and share data they co-generate and seeking to rebalance contractual power in data-sharing agreements. While these frameworks establish legal rights and obligations, this paper envisions a set of technical primitives that can help operationalize compliance and foster an ecosystem that embodies their spirit. 
